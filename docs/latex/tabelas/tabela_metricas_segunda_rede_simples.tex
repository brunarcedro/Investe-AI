% Tabelas Simples - Métricas da Segunda Rede Neural
% Para incluir em documento existente

% ============================================
% TABELA 1: Métricas Principais
% ============================================
\begin{table}[h!]
\centering
\caption{Métricas de desempenho da segunda rede neural (MLPRegressor)}
\label{tab:metricas_segunda_rede}
\begin{tabular}{lcc}
\toprule
\textbf{Métrica} & \textbf{Treinamento} & \textbf{Teste} \\
\midrule
MSE (Mean Squared Error) & 0.001212 & 0.003644 \\
MAE (Mean Absolute Error) & 0.025476 & 0.045389 \\
R² (Coeficiente de Determinação) & 0.6747 & 0.1998 \\
\midrule
\textbf{Amostras} & 400 & 100 \\
\bottomrule
\end{tabular}
\end{table}

% ============================================
% TABELA 2: Interpretação em Percentuais
% ============================================
\begin{table}[h!]
\centering
\caption{Interpretação das métricas em percentuais de alocação}
\label{tab:interpretacao_percentuais}
\begin{tabular}{lcl}
\toprule
\textbf{Métrica} & \textbf{Valor} & \textbf{Interpretação Prática} \\
\midrule
\multicolumn{3}{c}{\textit{Conjunto de Teste}} \\
\midrule
MSE & 0.36\% & Erro quadrático médio \\
MAE & 4.54\% & Erro médio de $\pm$4.54 p.p. por ativo \\
R² & 19.98\% & Explica 20\% da variação \\
\bottomrule
\end{tabular}
\end{table}
\begin{table}[h!]
\centering
\caption{Interpretação das métricas em percentuais de alocação}
\label{tab:interpretacao_percentuais}
\begin{tabular}{lcl}
\toprule
\textbf{Métrica} & \textbf{Valor} & \textbf{Interpretação Prática} \\
\midrule
\multicolumn{3}{c}{\textit{Conjunto de Teste}} \\
\midrule
MSE & 0.36\% & Erro quadrático médio \\
MAE & 4.54\% & Erro médio de $\pm$4.54 p.p. por ativo \\
R² & 19.98\% & Explica 20\% da variação \\
\bottomrule
\end{tabular}
\end{table}

% ============================================
% TABELA 3: Arquitetura da Rede
% ============================================
\begin{table}[h!]
\centering
\caption{Configuração da segunda rede neural}
\label{tab:arquitetura}
\begin{tabular}{ll}
\toprule
\textbf{Parâmetro} & \textbf{Valor} \\
\midrule
Tipo de Rede & MLPRegressor \\
Camadas Ocultas & (100, 50) neurônios \\
Função de Ativação & ReLU \\
Otimizador & Adam \\
Taxa de Aprendizado & 0.001 (adaptativa) \\
Regularização $\alpha$ & 0.001 \\
Early Stopping & Sim (15\% validação) \\
Máximo de Iterações & 1500 \\
\bottomrule
\end{tabular}
\end{table}

% ============================================
% INTERPRETAÇÃO RESUMIDA
% ============================================

\subsection*{Interpretação dos Resultados}

\textbf{MSE (Mean Squared Error):}
O erro quadrático médio de 0.003644 no teste (0.36\%) indica que os erros de alocação são pequenos, com penalização maior para erros grandes.

\textbf{MAE (Mean Absolute Error):}
O erro absoluto médio de 4.54\% significa que, em média, o modelo erra 4.54 pontos percentuais por classe de ativo. Por exemplo, se a alocação ideal é 30\% em ações, o modelo pode prever entre 25.46\% e 34.54\%.

\textbf{R² (Coeficiente de Determinação):}
\begin{itemize}
    \item \textbf{Treino:} R² = 67.47\% (bom desempenho)
    \item \textbf{Teste:} R² = 19.98\% (desempenho limitado)
    \item A diferença indica \textbf{overfitting} - o modelo memoriza padrões do treino mas não generaliza bem
\end{itemize}

\textbf{Diagnóstico:}
\begin{itemize}
    \item Overfitting evidenciado pelo R² de teste 70\% menor que o de treino
    \item MAE em teste (4.54\%) quase 2× maior que em treino (2.55\%)
    \item Causas prováveis: dataset pequeno (500 amostras), arquitetura complexa, dados sintéticos
\end{itemize}

\textbf{Avaliação para TCC:}
\begin{itemize}
    \item[$\checkmark$] Adequado como prova de conceito
    \item[$\checkmark$] MAE de 4.54\% é razoável para recomendação inicial
    \item[$\times$] R² baixo (20\%) limita capacidade preditiva
    \item[$\checkmark$] Demonstra conhecimento de métricas e arquitetura de redes neurais
\end{itemize}

\textbf{Melhorias sugeridas:}
Aumentar dataset, aplicar regularização mais forte, usar cross-validation, simplificar arquitetura para (50, 25) neurônios, validar com dados reais.
